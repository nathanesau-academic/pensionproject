\documentclass{report}

\usepackage{template}

\begin{document}\large{}

\begin{titlepage}
\color{bleu}

%\includegraphics[scale=0.27]{sfuLogo}

\vspace*{3cm} \Huge
\begin{leftbar}
{\normalfont\sffamily\Huge\bfseries\color{bleu}Actuarial Valuation \\ as at Dec 31, 2013}
\end{leftbar}


\vspace{15mm}
\begin{table}[ht]
\large
\begin{tabular}{p{5cm} p{5cm}}
\hspace{5mm}301154503 & \hspace{5mm} 301197568 \\
$\overline{\mbox{ \hspace{35mm} }}$ & $\overline{\mbox{ \hspace{35mm} }}$ \\ \mbox{ } Phoebe Zhang & \mbox{ } Nathan Esau\vspace{9mm} \\
\hspace{5mm} 301168614 & \hspace{5mm} 301177603 \\
$\overline{\mbox{ \hspace{35mm} }}$ & $\overline{\mbox{ \hspace{35mm} }}$ \\ \mbox{ } Layla Trummer & \mbox{ } Albee Hong\vspace{9mm} \\ 
\hspace{5mm} 301212599 & \hspace{5mm} 301217563 \\ 
$\overline{\mbox{ \hspace{35mm} }}$ & $\overline{\mbox{ \hspace{35mm} }}$ \\ \mbox{ } Jonathan Royalty & \mbox{ } Tom Dodge \vspace{9mm} \\ 
\end{tabular}
\end{table}

\vspace{3mm}
{\color{black}\large
November 25th, 2014}

\end{titlepage}

\tableofcontents
\large

\pagebreak
\mychapter{0}{Executive Summary}
An actuarial valuation has been prepared for Simon Fraser University as at December 31, 2013 for the purpose of monitoring Simon Fraser University's funding position. This section provides an overview of the important results and the key inputs to the valuation process. The next actuarial valuation for the purposes of developing funding requirement must be performed no later than December 31, 2016.

\section{Summary of Main Results}
\begin{table}[ht]
\bgroup
\def\arraystretch{1.5}
\begin{tabular}{l  l  l  l  l}
\hline
& \textbf{Dec 31, 2012} &  & \textbf{Dec 31, 2013} & \\ \hline
& Going-Concern & Solvency & Going-Concern & Solvency \\ \hline
Assets & 9,571,300 & 9,571,300  & 10,711,300 & 10,386,300 \\ \hline
Liabilities & 9,921,600 & 11,002,700 & 10,011,100 & 10,895,800 \\ \hline
Surplus(Deficit) & (350,300) & (1,731,400) & 700,200 & (509,500) \\ \hline
Funded/Solvency Ratio & 96\% & 87\% & 107\% & 95\% \\ \hline
\end{tabular}
\egroup
\end{table}

\section{Contribution Rate}

The contribution rate in this valuation and the previous one are shown below:

\begin{table}[ht]
\bgroup
\def\arraystretch{1.5}
\begin{tabular}{p{6cm} p{3cm} p{3cm}}
\hline
 & \textbf{Dec 31, 2013} &  \textbf{Dec 31, 2012} \\ \hline
Employer Contributions & 230,700 & 306,200 \\ \hline
Expected Salary for Next Year & 1,646,000 & 2,176,200 \\ \hline
Contribution Rate & 14.0\% & 14.1\% \\ \hline
\end{tabular}
\egroup
\end{table}

\vspace{3mm}
The minimum and maximum contributions for the following year are:

\begin{table}[ht]
\bgroup
\def\arraystretch{1.5}
\begin{tabular}{p{6cm} p{3cm} p{3cm}}
\hline
 & \textbf{Minimum} &  \textbf{Maximum} \\ \hline
Employer Contributions & 230,700 & 230,700 \\ \hline
Solvency Special Payment & 109,300 & 509,500 \\ \hline
Total Payment & 340,000 & 740,200 \\ \hline
\end{tabular}
\egroup
\end{table}

\pagebreak
\section{Basic Membership Information}

As at the valuation date, the basic membership information is as follows:

\begin{table}[ht]
\bgroup
\def\arraystretch{1.5}
\begin{tabular}{p{5.2cm} p{1.8cm} p{2.5cm} p{2.2cm} p{2.2cm}}
\hline
& Active Members & Deferred Vested Members & Pensioners and Survivors & Total \\ \hline
Going-Concern Liability (\%) & 63.54\% & 3.78\% & 32.68\% & 100\% \\ \hline 
Solvency Liability (\%) &  61.69\% & 4.57\% & 33.74\% & 100\% \\ \hline
Members as at Dec 31, 2013 & 41 & 6 & 14 & 61 \\ \hline 
\end{tabular}
\egroup
\end{table}

There were also four members who died or were paid-out during the year:

\section{Key Assumptions}
The principal assumptions to which the valuation results are most sensitive are outlined in the following table:

\begin{table}[ht]
\bgroup
\def\arraystretch{1.5}
\begin{tabular}{l  l  l  l  l}
\hline
& \textbf{Dec 31, 2012} &  & \textbf{Dec 31, 2013} & \\ \hline
& Going-Concern & Solvency & Going-Concern & Solvency \\ \hline
Discount Rate & 4.75\% & 3\% (AP), 2.5\% (CV)  & 5\% & 3.5\% (AP), 3.25\% (CV) \\ \hline
Salary Scale & 4\% & - & 4.25\% & - \\ \hline
Mortality & UP94 (to 2020) & UP94 (to 2020) & UP94 (to 2015) & UP94 (to 2015) \\ \hline
Retirement Age & 60 & 55 (if not eligible) & 65 & 55 (if not eligible) \\ \hline
\end{tabular}
\egroup
\end{table}

\mychapter{1}{Introduction}

\color{black}
At the request of Simon Fraser University, we have conducted an actuarial valuation of a small corporate pension plan as at December 31, 2013. We are pleased to present the results of the valuation.

\section{Purpose} 

The purpose of this valuation is to determine:
\begin{itemize}
\item The funding position on going-concern and solvency bases as  at December 31, 2013 
\item Reconciliation of the going-concern funding position at the last actuarial valuation (December 31, 2012) with the going-concern funding position at December 31, 2013, identifying gain and loss elements by sources
\item The normal cost for 2014
\item The minimum contribution requirements for 2014
\end{itemize}

\vspace{3mm}
The information in this report is intended for the internal use of Simon Fraser University and its authorized parties, and for filing with the Canadian Revenue Agency, in connection with our actuarial valuation of the plan. This report is not intended or suitable for any other purpose.

\vspace{3mm}
In accordance with pension benefits legislation, the next actuarial valuation of the plan will be required with an effective date no later than December 31, 2016, or as at the date of an earlier amendment to the plan.

\section{Summary of Changes Since the Last Valuation}

Since the last valuation at December 31, 2012, we note that the following have occurred:

\begin{itemize}
\item Plan Provision

We have used the same retirement age, normal pension form, and methods for termination benefit and death benefit as were used for the previous valuation, except for the following:

\vspace{3mm}
\bgroup
\normalsize
\def\arraystretch{1.5}
\begin{tabular}{p{4.5cm} p{5cm}  p{5cm}}
\hline
& Current Valuation & Previous Valuation \\ \hline
Pension Formula & 1.3\% of FAE3 up to FAYMPE3 + 2.0\% of FAE3 in excess of FAYMPE3 per year of service & 1.3\% of FAE5 up to FAYMPE5 + 2.0\% of FAE5 in excess of FAYMPE5 per year of service \\ \hline
Early Retirement Reduction & Actuarial equivalent to age 65 & 3\% per year to age 65 \\ \hline
\end{tabular}
\egroup

\vspace{3mm}
\item Going-concern actuarial methods and assumptions:

Various going-concern assumptions have been changed to reflect changes in future expectations as well as to realign the margin contained in the assumptions such that the margin is predominantly reflected in the discount rate assumption.

\vspace{3mm}
\bgroup
\normalsize
\def\arraystretch{1.5}
\begin{tabular}{p{4.5cm} p{5cm}  p{5cm}}
\hline
& Current Valuation & Previous Valuation \\ \hline
Discount Rate & 4.75\% & 5.00\% \\ \hline
Salary Increase & 4.00\% & 4.25\% \\ \hline
Mortality Rates & UP94 (to 2020) & UP94 (to 2015) \\ \hline
Retirement Age (Active) & Age 60 & Age 65 \\ \hline
Retirement Age (Deferred) & Age 60 & Age 65 \\ \hline
\end{tabular}
\egroup

\vspace{3mm}
\item Solvency actuarial methods and assumptions:

Solvency discount rate and mortality rates have been changed to reflect future expectations. Also, the amount of wind-up expense has been increased.

\vspace{3mm}
\bgroup
\normalsize
\def\arraystretch{1.5}
\begin{tabular}{p{4.5cm} p{5cm}  p{5cm}}
\hline
& Current Valuation & Previous Valuation \\ \hline
Discount Rates & 3.25\% (CV), 3.50\% (AP) & 2.50\% (CV), 3.00\% (AP) \\ \hline
Mortality Rates & UP94 (to 2020) & UP94 (to 2015) \\ \hline
Wind-up Expense & \$325,000 & \$300,000 \\ \hline
\end{tabular}
\egroup

\vspace{3mm}
A more detailed summary of the going-concern and solvency methods and assumptions is provided in Appendices C and D, respectively.
\end{itemize}

\section{Subsequent Events}
As of the date of this report, we have not been made aware of any subsequent events which would have an effect on the results of this valuation. However, the following points should be noted in this regard:
\begin{itemize}
\item Actual experience deviating from expected, from the valuation date to the date of this report, will result in gains or losses
\item To the best of our knowledge, the results contained in this report are based on the regulatory and legal environment in effect at the date of this report and do not take into consideration any potential changes that are currently the subject of debate, review and/or court appeal. To the extent that actual changes in the regulatory and legal environment transpire, any financial impact on Simon Fraser University as a result of such changes will be reflected in future valuations
\end{itemize}

\mychapter{2}{Valuation Results - Going-Concern}

\section{Financial Position}
A going-concern valuation compares the relationship between the value of plan assets and the present value of expected future benefit cash flows in respect of accrued service, assuming the plan will be maintained indefinitely.

The results of the current valuation, compared with that from the previous valuation are summarized as follows:

\vspace{3mm}
\bgroup
\normalsize
\def\arraystretch{1.5}
\begin{tabular}{p{5.0cm} p{4.5cm}  p{4.5cm}}
\hline
& \textbf{December 31, 2013 (\$)} & \textbf{December 31, 2012 (\$)} \\ \hline
\textbf{Assets} & 10,711,300 & \$9,571,300 \\ 
\textbf{Liability} \\ 
 \hspace{3mm} Active Members & 6,361,200 & 8,311,200 \\ 
\vspace{-5mm} \hspace{3mm} Terminated Vested Members & \vspace{-5mm} 378,200 & \vspace{-5mm} 286,000 \\ 
\vspace{-5mm}  \hspace{3mm} Pensioners and Survivors & \vspace{-5mm} 3,271,700 & \vspace{-5mm} 1,324,400 \\ 
\textbf{Total Liability} & 10,011,100 & 9,921,600 \\
\hline 
\textbf{Surplus/Deficit} & 700,200 & (350,300) \\ \hline
\textbf{Funded Ratio} & 100\% &  96\% \\ \hline
\end{tabular}
\egroup

\section{Change in Going-Concern Financial Position}
\bgroup
\normalsize
\def\arraystretch{1.5}
\begin{longtable}[l]{p{6.5cm} p{6cm}}
\uline{\textbf{Reconciliation of Financial Status }}&  \\
\textbf{As at December 31, 2012} & \textbf{(350,300)} \\ \\
Interest & (17,500) \\
Investment & (699,800) \\ 
Current Service Cost & (92,400) \\ 
Special Payment & 1,782,500 \\ 
Salary Experience & 41,900 \\ 
Retirement Experience & 80,800 \\ 
Termination Experience & 99,600 \\
Mortality Experience & 356,000 \\ \\
\uline{Change in Plan Provisions} \\
Final Average Earnings & (242,500) \\ 
Early Retirement Reduction & 60,400 \\ 
Discount Rate & (313,000) \\
Salary Increase & 84,400 \\ 
Mortality & (89,900) \\ 
Miscellaneous & (100) \\ \\
\textbf{As at December 31, 2013} & \textbf{700,200}
\end{longtable}
\egroup

\section{Commentary of Gain 
and Loss}

At the end of 2013 we have a set of assumptions that are different than from the last valuation. When restoring the going-concern assumptions to what they were in the last valuation, we used the following order:
\begin{itemize}
\item Discount Rate
\item Mortality
\item Salary (restoring previous salary scale)
\item Retirement Age
\item Pension Formula (changing formula to incorporate FAE3 and FAYMPE3)
\item Restore the 2013 salaries and 2014 YMPE to figures used in the previous valuation
\end{itemize}

\vspace{3mm}
Depending on the member category, some of these restorative changes were not necessary. For example, we did not need to restore salary or retirement age assumptions for pensioners and survivors.

\vspace{3mm}
There were active two members who received a lump sum and left the plan. This loss is reflected in current service cost and the associated gain is reflected in termination experience. 

\vspace{5mm}
The gain associated with the change in the retirement age assumption is reflected in the early retirement reduction. 

\vspace{3mm}
The final average earnings loss also accounts for the change in final average YMPE

\vspace{3mm}
Restoring the 2013 salaries and 2014 YMPE figures was accounted for in the salary experience.

\vspace{3mm}
The intuition behind the gain and losses is outlined below,

\begin{itemize}
\item \textbf{Interest} resulted in a loss since beginning of year liabilities exceeded beginning of year assets
\item \textbf{Investment} resulted in a loss since rate of return of -1.84\% was much lower than the expected return of 5\% 
\item \textbf{Current Service Cost} resulted in a loss since the contributions were much lower than required under the projected unit credit method and new entrants entered the plan, requiring new funding
\item \textbf{Special Payment(s)} were made by the employer to cover a solvency deficit resulting in an asset gain
\item \textbf{Salary Experience} resulted in a gain since salaries increased less than expected for every member 
\item \textbf{Retirement Experience} resulted in a gain due to a large proportion of active members retiring early, receiving lower annual pensions than expected. Late retirements also resulted in gains
\item \textbf{Termination Experience} resulted in a gain due to the two active members who unexpectedly received lump sum payouts
\item \textbf{Mortality Experience} resulted in a gain due to a large proportion of pensioners who died during the year
\item \textbf{Final Average  Earnings} resulted in a loss due to a three year average being used in this valuation leading to higher than expected annual pensions
\item \textbf{Early Retirement Reduction} is made up of two components. The first is a change in retirement age assumption, which resulted in a gain (from members accruing lower pensions). The second is a loss from changing early retirement reduction to 3\% per year (more generous than actuarial equivalent). The salary scale is higher than the reduction rate of 3\% which is one reason for the overall gain
\item \textbf{Discount Rate} assumption decreasing resulted in accrued liabilities having higher present values leading to a loss
\item \textbf{Salary Increase} assumption decreasing resulted in the projected benefits decreasing, resulting in a gain
\item \textbf{Mortality} improvements from the new UP94 table lead to a loss
\end{itemize}

\section{Current Service Cost}
The current service cost is an estimate of the present value of the additional expected future benefit cash flows in respect of pensionable service that will accrue after the valuation date, assuming the plan will be maintained indefinitely.

\vspace{3mm}
The current service cost during the year following the valuation date compared with the corresponding value determined in the previous valuation is as follows:

\vspace{3mm}
\bgroup
\normalsize
\def\arraystretch{1.5}
\begin{tabular}{p{6.5cm} p{5cm}}
\hline
& December 31, 2013 \\ \hline
Actuarial present value of service cost at the middle of the year & 193,257 \\ \hline
Service cost for original active members & (265,911) \\ \hline
Service cost for new entrants & (19,771) \\ \hline
Total current service cost & (92,400) \\ \hline
\end{tabular}
\egroup

\section{Discount Rate Sensitivity}

The following table summarizes the effects on going-concern liabilities using a discount rate which is 1\% lower than that used in the valuation.

\vspace{3mm}
\bgroup
\normalsize
\def\arraystretch{1.5}
\begin{tabular}{p{6.5cm} p{3cm} p{4.5cm}}
\hline
& Valuation Basis & Reduce Discount Rate by 1\% \\ \hline
Total Going-Concern Liabilities & 10,011,100 & 11,440,283 \\ \hline
\end{tabular}
\egroup

\mychapter{3}{Valuation Results - Solvency}

\section{Solvency Financial Position}

The solvency valuation is required by the \textit{Income Tax Act} for the purpose of providing an assessment of a company's financial position at the valuation date on the premise that the company's obligations are settled on the valuation date for all members. %The \textit{Income Tax Act} does not require funding based on the solvency valuation results. However, funding the solvency liability is recommended.

\vspace{3mm}
The financial position of the company on the solvency basis is measured by comparing the value of the assets, reduced by an allowance for estimated wind-up expenses, with the actuarial liability for benefits earned for service up to the valuation date determined on the assumption the company is terminated on the valuation date, with immediate settlement of liabilities.

\vspace{3mm}
On the basis of the company's provisions, membership date, solvency assumptions and methods and asset information described in the Appendices, as well as the requirement of the \textit{Income Tax Act} the solvency financial position of the company as at December 31, 2013 is shown in the following table. The solvency financial position of the company as at December 31, 2012 is shown for comparison purposes.

\vspace{3mm}
\bgroup
\normalsize
\def\arraystretch{1.5}
\begin{tabular}{p{5.0cm} p{4.5cm}  p{4.5cm}}
\hline
& \textbf{December 31, 2013 (\$)} & \textbf{December 31, 2012 (\$)} \\ \hline
\textbf{Assets} & \$10,386,300 & \$9,271,300 \\ 
\textbf{Liability} \\ 
 \hspace{3mm} Active Members & 6,721,500 & 9,019,200 \\ 
\vspace{-5mm}  \hspace{3mm} Terminated Vested Members & \vspace{-5mm} 497,800 & \vspace{-5mm} 417,100 \\ 
\vspace{-5mm}  \hspace{3mm} Pensioners and Survivors & \vspace{-5mm} 3,676,500 & \vspace{-5mm} 1,566,400 \\ 
\textbf{Total Liability} & 10,895,800 & 11,002,700 \\
\hline 
\textbf{Surplus/Deficit} & (509,000) & (1,731,400) \\ \hline
\textbf{Funded Ratio} & 95\% &  84\% \\ \hline
\end{tabular}
\egroup

\section{Impact of Plan Wind-up}

In our opinion, the value of the plan's assets would be less than its liabilities if the plan were to be wound up on the valuation date.

\vspace{3mm}
Specifically, actuarial liabilities would exceed the market value of plan assets available to provide benefits by \$509,500. This calculations includes a provision of \$325,000 for plan termination expenses that might be payable from the pension fund if the plan were wound up.

\vspace{3mm}
Since the solvency valuation resulted in a deficit, this must be amortized over five years. The discount rate for these payments was calculated as a weighted average of the commuted value and annuity purchase solvency liabilities. This led to a blended discount rate of 2.82\%.

\section{Solvency Incremental Cost}

The solvency incremental cost is an estimate of the present value of the projected solvency liabilities three years after the valuation date, adjusted for the benefit payments expected to be made in that period. It was also assumed that membership does not change.

The solvency incremental cost determined in this valuation is as follows,

\bgroup
\normalsize
\def\arraystretch{1.5}
\begin{longtable}[l]{p{10cm} p{6cm}}
Total Solvency Liabilities at Valuation Date & \hspace{3mm} \$ \hspace{5mm} 10,895,000 \\
Incremental Cost &  \hspace{7mm} \uline{\hspace{6mm} 1,051,300 } \\ 
Total (with Incremental Cost) & \hspace{12mm} \textbf{11,946,300 } \\
\end{longtable}
\egroup

\section{Discount Rate Sensitivity}

For our solvency valuation, it was assumed that active members would receive a portion of their benefit in the form of a commuted value and a portion in the form of an annuity purchase. The proportion of these components is different for active members who are eligible to retire than for active members who are not eligible to retire. Furthermore, each of these components was valued at a different discount rate. All other members in the plan are assumed to receive an annuity purchase. The details on these assumptions are provided in Appendix D.

\vspace{3mm}
If we use a discount rate that is 1\% lower for each individual component (commuted value and annuity purchase) the total solvency liability increases by the amount shown in the following table.

\vspace{3mm}
\bgroup
\normalsize
\def\arraystretch{1.5}
\begin{tabular}{p{6.5cm} p{3cm} p{4.5cm}}
\hline
& Valuation Basis & Reduce Discount Rate by 1\% \\ \hline
Total Solvency Liabilities & 10,895,800 & 12,369,200 \\ \hline
\end{tabular}
\egroup


\mychapter{4}{Actuarial Certificate}

Actuarial Opinion and Certification for \textit{Simon Fraser University}

\section{Opinion}

This actuarial certification forms an integral part of the actuarial valuation report for SFU as at December 31, 2013. We confirm that we have prepared an actuarial valuation of SFU as at December 31, 2013 for the purposes outlined in the Introduction of this report and consequently,

\vspace{8mm}
\begin{center}{\bf\Large\color{bleu}We hereby certify that, in our opinion:}\end{center}

\vspace{3mm}
\begin{enumerate}
\item[\color{bleu}\LARGE\bf1. ] With respect to the purposes of determining SFU's financial position on a going-concern basis:
\begin{itemize}
\item SFU has a going-concern surplus (excess of assets over liabilities) of \$700,200 as at December 31, 2013 based on total assets of \$10,711,300 and total liabilities of \$10,011,100
\item There is no excess surplus as defined by Section 147.2(2) of the \textit{Income Tax Act} as at December 31, 2013 
\end{itemize}
\item[\color{bleu}\LARGE\bf2. ] With respect to the  purpose of determining SFU's financial position on a solvency basis:
\begin{itemize}
\item SFU has a solvency deficit of \$509,500 as at December 31, 2013, determined as solvency assets of \$10,386,300 less solvency liabilities of \$10,895,800
\item The solvency ratio is 95\% as at December 31, 2013
\item The liabilities of SFU would exceed SFU's assets by \$509,500 if SFU was terminated and wound-up as at December 31, 2013
\end{itemize}
\item[\color{bleu}\LARGE\bf3. ] If a funding recommendation was formulated based on the December 31, 2013 valuation results, the contribution rates that would be applicable to SFU for years after 2013 are as follows:
\begin{itemize}
\item SFU's total current service cost would be determined as 14\% of aggregate member contributory earnings
\item The minimum and maximum employer contribution amounts for 2014 are \$340,000 and \$740,200 respectively
\end{itemize}
\item[\color{bleu}\LARGE\bf4. ] For the purposes of the valuation:
\begin{itemize}
\item The data of which this valuation is based are sufficient and reliable;
\item The assumptions used are appropriate; and
\item The actuarial cost methods and the asset valuation methods used are appropriate
\item[\color{bleu}\LARGE\bf5. ] This report and its associated work have been prepared, and our opinion given, in accordance with accepted actuarial practice in Canada and in compliance with the requirements outlined in subparagraphs 147.2(2)(a)(iii) and (iv) of the \textit{Income Tax Act}
\item[\color{bleu}\LARGE\bf6. ] Notwithstanding the above certifications, emerging experience differing from the assumptions will result in gains or losses that will be revealed in subsequent valuations
\end{itemize}
\end{enumerate}

\vspace{8mm}
Original signed by,
\vspace{8mm}

\begin{table}[ht]
\large
\begin{tabular}{p{8cm} p{8cm}}
\hspace{5mm} $\mathnormal{Phoebe}$ $\mathnormal{Zhang}$ & \hspace{5mm} $\mathnormal{Nathan}$ $\mathnormal{Esau}$ \\
$\overline{\mbox{ \hspace{60mm} }}$ & $\overline{\mbox{ \hspace{60mm} }}$ \\ \mbox{ } Phoebe Zhang, Actuarial Analyst & \mbox{ } Nathan Esau, Actuarial Analyst \vspace{9mm} \\ 
\hspace{5mm} $\mathnormal{Layla}$ $\mathnormal{Trummer}$ & \hspace{5mm} $\mathnormal{Albee}$ $\mathnormal{Hong}$ \\
$\overline{\mbox{ \hspace{60mm} }}$ & $\overline{\mbox{ \hspace{60mm} }}$ \\ \mbox{ } Layla Trummer, Actuarial Analyst & \mbox{ } Albee Hong, Actuarial Analyst \vspace{9mm} \\ 
\hspace{5mm} $\mathnormal{Jonathan}$ $\mathnormal{Royalty}$ & \hspace{5mm} $\mathnormal{Tom}$ $\mathnormal{Dodge}$ \\ 
$\overline{\mbox{ \hspace{60mm} }}$ & $\overline{\mbox{ \hspace{60mm} }}$ \\ \mbox{ } Jonathan Royalty, Actuarial Analyst & \mbox{ } Tom Dodge, Actuarial Analyst \vspace{9mm} \\ 
\end{tabular}
\end{table}

\vspace{3mm}
November 25th, 2014

\mychapter{5}{Appendix A: Plan Assets}

The pension fund is held in trust by Simon Fraser Investment Management Corporation in this valuation and the last valuation. The asset data in this valuation relies upon the plan's financial statements prepared by Simon Fraser Investment Management Corporation. Plan assets were valued based on Market Value of assets.

\section{Asset Allocation}
The following is a summary of the composition of SFU's trust fund asset by asset type as reported in the financial statements as at December 31, 2013

\bgroup
\normalsize
\def\arraystretch{1.5}
\begin{longtable}[l]{p{8.5cm} p{6cm}}
Bonds & \$ \hspace{5mm} 5,827,000 \\
Equities & \hspace{8mm} 3,352,300 \\
Cash &  \hspace{1mm} \uline{\hspace{6mm} 1,532,000} \\ 
Total Investment Assets & \hspace{8mm} \textbf{9,571,300} \\
\end{longtable}
\egroup

\section{Reconciliation of Market Value of Assets}

The pension fund transactions since the last valuation are summarized below. The following table reconciles the opening and closing market value of the fund over the past year.

\vspace{3mm}
\bgroup
\normalsize
\def\arraystretch{1.5}
\begin{tabular}{p{8.5cm} p{4.5cm}}
\hline
\textbf{Market Value of Assets as at December 31, 2012 }&  \hspace{5mm} \$ \hspace{5mm} 9,571,300 \\ \hline 
Contribution: \\ 
 \hspace{3mm} Employer current service cost contributions & \hspace{14mm} 188,600 \\ 
\vspace{-5mm} \hspace{3mm} Special Payments & \vspace{-5mm}  \hspace{14mm} 1,739,500 \\  
Benefits: \\ 
 \hspace{3mm} Employer current service cost contributions & \hspace{14mm} (414,200) \\ 
\vspace{-5mm}  \hspace{3mm} Special Payments & \vspace{-5mm} \hspace{14mm} (185,500) \\  
Investment Return & \hspace{14mm} (188,400) \\ \hline
\textbf{Market Value of Assets as at December 31, 2013} & \hspace{14mm} 10,711,300 \\ \hline
Rate of return & \hspace{14mm} -1.84\% \\ \hline
\end{tabular}
\egroup

\vspace{3mm}
In the rate of return calculation, we have assumed that cash flows are made in the middle of the year.

\section{Investment Income}

The change in the going-concern financial position section of this report uses the gain or loss from investment income as a balancing account for assets. Under this method, the gain or loss from investment income was -\$699,800. 

\vspace{3mm}
To obtain this result, the \textit{gain/loss accounts} were brought forward to the end of the year using compound interest at 5\%. If the timing of cash flows was unknown, it was assumed that cash flows occurred in the middle of the year.

\vspace{3mm}
Another way to calculate investment income    is to bring forward each \textit{asset} account with interest. Doing so, we found that an end year asset of \$11,412,000 was  expected. However, our end of year asset was \$10,711,300, meaning that investment was -\$699,800 lower than expected.

\mychapter{6}{Appendix B: Membership Data}

\section{Source of Membership Data}

This actuarial valuation is based on membership data provided by Simon Fraser University as at December 31, 2013. Various tests on the membership data were conducted to ensure its validity. Tests performed included the following:

\begin{itemize}
\item Membership reconciliation with prior valuation data, which is presented below
\item Comparison of changes in salaries, years of membership and credited service between the 2012 and 2013 data
\item Comparison of pensions paid to retirees and lump sum benefits paid following termination of employment or death to amounts contained in the asset data
\item Validation with APS of all deviations observed in information compared to data provides for the previous actuarial valuation and adjustments made where necessary
\end{itemize}

\vspace{3mm}
For calculating years and membership age, we assumed that there are 365.25 days in a year. We then rounded ages and service to the nearest year. In cases where the member's age and service in 2013 were unchanged from the 2012 values (due to our rounding method) we manually corrected the member's age.

\section{Membership Reconciliation}

\begin{table}[ht]
\bgroup
\def\arraystretch{1.5}
\begin{tabular}{p{4.7cm} p{1.5cm} p{2.5cm} p{4.2cm} p{1.0cm}}
\hline
& Active & Deferred Vested & Pensioners and Survivors & Total \\ \hline
\textbf{As at December 31, 2012} & 45 & 5 & 10 & 60 \\ \hline
New Members & 5 & - & - & 5 \\ \hline
Deferred & (2) & 2 & - & - \\ \hline
Retirements & (5) & (1) & 6 & - \\ \hline
Termination & (1) & - & - & (1) \\ \hline
Death & (1) & - & (2) & (3) \\ \hline
\textbf{As at December 31, 2013} & 41 & 6 & 14 & 61 \\ \hline
\end{tabular}
\egroup
\end{table}

\pagebreak
\section{Membership Summary}

\textbf{Active Members} 

\vspace{3mm}
\bgroup
\normalsize
\def\arraystretch{1.5}
\begin{tabular}{p{4.0cm} p{2.5cm} p{4.0cm} p{4.0cm}}
\hline
& & \textbf{December 31, 2013 } & \textbf{December 31, 2012 } \\ \hline
Number \\
& Male & 27 & 30 \\ 
& Female & 14 & 15 \\ 
& \textbf{Total} & 41 & 45 \\ 
Average Age \\
& Male & 48.6 & 49.5 \\ 
& Female & 46.1 & 47.3 \\
& \textbf{Total} & 47.8 & 48.8 \\
Average Service  \\
& Male & 16.9 & 19.7 \\ 
& Female & 15.3 & 17.9 \\
& \textbf{Total} & 16.3 & 19.1 \\
Average Salary  \\
& Male & 50,113.60 & 52,593.80 \\
& Female & 47,391.70 & 50,907.70 \\
& \textbf{Total} & 49,184.20 & 52,031.80 \\ \hline
\end{tabular}
\egroup

\vspace{8mm}
\textbf{Deferred Members} 

\vspace{3mm}
\bgroup
\normalsize
\def\arraystretch{1.5}
\begin{tabular}{p{4.0cm} p{2.5cm} p{4.0cm} p{4.0cm}}
\hline
& & \textbf{December 31, 2013 } & \textbf{December 31, 2012 } \\ \hline
Number \\
& Male & 4 & 3 \\ 
& Female & 2 & 2 \\ 
& \textbf{Total} & 6 & 5 \\ 
Average Age \\
& Male & 47.5 & 52.7 \\ 
& Female & 49.0 & 48.0 \\
& \textbf{Total} & 48.0 & 50.8 \\
Average Annual Pension \\
& Male & 6,900.70 & 9,002.20 \\
& Female & 5,880.90 & 6,541.90 \\
& \textbf{Total} & 6,560.80 & 8,018.10 \\ \hline
\end{tabular}
\egroup

\textbf{Pensioners and Survivors}

\vspace{3mm}
\bgroup
\normalsize
\def\arraystretch{1.5}
\begin{tabular}{p{4.0cm} p{2.5cm} p{4.0cm} p{4.0cm}}
\hline
& & \textbf{December 31, 2013 } & \textbf{December 31, 2012 } \\ \hline
Number \\
& Male & 7 & 5 \\ 
& Female & 7 & 5 \\ 
& \textbf{Total} & 14 & 10 \\ 
Average Age \\
& Male & 63.7 & 73.4 \\ 
& Female & 65.4 & 65.4 \\
& \textbf{Total} & 64.6 & 69.4 \\
Average Annual Pension \\
& Male & 21,345.00 & 16,597.10 \\
& Female & 15,134.40 & 10,816.20 \\
& \textbf{Total} & 18,239.70 & 13,706.60 \\ \hline
\end{tabular}
\egroup

\mychapter{7}{Appendix C: Methods and Assumptions -  Going-Concern}

A member's entitlements under a pension plan are generally funded during the period over which service is accrued by the member. In other words, the costs of each member's benefits are allocated in some fashion over each member's service. An actuarial valuation provides an assessment of the extent to which allocations relating to periods prior to a valuation date (often referred to as the actuarial liabilities) are covered by the plan's assets.

\vspace{3mm}
The going-concern valuation provides an assessment of a pension plan on the premise that the plan continues on into the future indefinitely. In order to prepare a going-concern valuation, two important elements need to be established:

\begin{itemize}
\item Going-concern assumptions in respect of future events upon which a plan's benefits are contingent; and
\item Going-concern methods which effectively determine the way in which a plan's costs will be allocated over the members' service
\end{itemize}

\vspace{3mm}
Together, the going-concern assumptions and methods provide a basis from which a plan's cost can be estimated and also help establish an orderly program for meeting the ultimate cost of a plan. The true cost of a plan, however, will emerge only as experience develops, investment earnings are received, and benefit payments are made. If a shortfall exists on the going-concern basis, this is amortized over 15 years. If, on the other hand a funded ratio of over 125\% develops on the going-concern basis, contributions can only be made if a solvency shortfall exists.

\vspace{3mm}
This appendix summarizes the going-concern assumptions and methods that have been adopted for the going-concern valuation of Simon Fraser University at the valuation date. It is important to note that these assumptions and methods should be reviewed periodically to ensure that they adequately reflect the experience of Simon Fraser University and continue to satisfy Simon Fraser University's funding objectives. For purposes of this valuation, the going-concern methods and assumptions were reviewed and changes as indicated below were made.

\section{Margins}
Margin for conservatism or provisions for adverse deviation have been built into the going-concern assumptions where appropriate. These margins are aimed at reducing the potential adverse effect of the uncertainty inherent in the going-concern assumptions. If the future unfolds in accordance with what are considered to be best estimate assumptions (that is, assumptions with no such margins), then the margin built into the going-concern assumptions will be released into surplus. These margins have been discussed with the Board, and the Board has expressed their comfort with the margin level.

\section{Going-Concern Assumptions}

\vspace{3mm}
\bgroup
\normalsize
\def\arraystretch{1.5}
\begin{tabular}{p{4.5cm} p{5cm}  p{5cm}}
\hline
& \textbf{December 31, 2013 }& \textbf{December 31, 2012}  \\ \hline
Mortality Rates & UP94 (to 2020) & UP94 (to 2015) \\ \hline
Retirement Age (Active) & Age 60 & Age 65 \\ \hline
Retirement Age (Deferred) & Age 60 & Age 65 \\ \hline
Discount Rate & 4.75\% & 5.00\% \\ \hline
Salary Increase & 4.00\% & 4.25\% \\ \hline
YMPE Increase & 3.00\% & 3.00\% \\ \hline
\end{tabular}
\egroup

\vspace{3mm}
Further detail concerning these assumptions is summarized below.

\vspace{3mm}
\textbf{Pre-Retirement Decrements}

It was assumed that pre-retirement decrements were not deemed to be material for the purposes of this valuation.

\vspace{3mm}
\textbf{Mortality Rates}

Benefits paid from Simon Fraser University in respect of a particular member are contingent to a very large degree on the survival of the member. For example:
\begin{itemize}
\item If an active member dies prior to retirement, a lump sum payment is triggered
\item A pension is paid to a pensioner only while the pension is alive
\end{itemize}

\vspace{3mm}
Consequently, an assumption has been made regarding the survival of members into the future.\

\vspace{3mm}
For this valuation, gender-distinct mortality rates have been assumed to be in accordance with the Uninsured Pensioner 1994 mortality table with mortality improvement projected to 2020. The same assumption was used in the previous valuation, but it was projected to 2015. In the valuation this year, the mortality is lower due to the additional 5 year improvement projection. 

\vspace{3mm}
The mortality rates as at both valuation dates are considered to be best estimate.

\vspace{3mm}
\textbf{Retirement} 

A member's benefit entitlement under Simon Fraser University is dependent on when the member decides to commence, or is deemed to commence, to receive a pension from Simon Fraser University (referred to as retirement from Simon Fraser University). Accordingly, an assumption with respect to when a member is expected to retire from Simon Fraser University has been made.

\vspace{3mm}
For this valuation, both active and deferred vested members are assumed to retire at age 60.

\vspace{3mm}
For the previous valuation, both active and deferred vested members are assumed to retire at age 65, which is the age that maximizes the pension value for such members.

\vspace{3mm}
\textbf{Future Simon Fraser University Membership for Funding the Unfunded Liability}

Simon Fraser University's unfunded liabilities are amortized as a level percentage of contributory payroll. For purposes of determining the applicable contribution rates, it has again been assumed that future new entrants will keep the active Simon Fraser University membership stable for the following valuation date, and that the total contributory pensionable earnings of these active members will increase at the assumed rate of general wage increases for Simon Fraser University members.

\vspace{3mm}
If the Simon Fraser University salary base were to grow more rapidly than assumed, existing unfunded liabilities might be eliminated sooner than assumed. Conversely, if the Simon Fraser University salary base were to grow less rapidly than assumed, contribution increases may be required in order to be able to eliminate existing unfunded liabilities within the required time frame.

\vspace{3mm}
\textbf{Discount Rate}

The actuarial present value of a future stream of benefit payments represents an estimate of the assets required at the valuation date that, together with future expected investment income, will be sufficient to provide for the future benefit stream. Therefore, in calculating actuarial present values, it is necessary to make an assumption with respect to the future expected investment returns that will be earned on Simon Fraser University's assets. This future investment rate of return is called the discount rate.

\vspace{3mm}
In selecting the going-concern discount rate assumption, the following factors are typically taken into consideration:

\begin{itemize}
\item Estimated returns for each major asset class consistent with market conditions on the valuation date and the target asset mix specified in the Plan's investment policy
\item Additional returns assumed to be achievable due to active equity management equal to the fees related to active equity management. Such fees were determined by the difference between the provision for total investment expenses and the hypothetical fees that would be incurred for passive management of all assets
\item Implicit provision for investment expenses determined based on historical expenses paid from the fund; and
\item A margin for adverse deviations
\end{itemize}

\vspace{3mm}
The discount rate was developed as follows:

\bgroup
\normalsize
\def\arraystretch{1.5}
\begin{longtable}[l]{p{8.5cm} p{6cm}}
Gross Rate of Return & \hspace{12mm} 6.44\% \\
Active Management & \hspace{12mm} 0.43\% \\
Administration Expenses & \hspace{11mm} (0.40\%) \\
Investment Expenses & \hspace{11mm} (0.68\%) \\
Margin for Adverse Deviation & \uline{\hspace{11mm} (1.04\%)} \\ 
Net Discount Rate & \hspace{12mm} 4.75\% \\
\end{longtable}
\egroup

\vspace{3mm}
A discount rate of  5\% was used in the previous valuation.

\vspace{3mm}
\textbf{Increases in Salaries}

As the benefits paid to a member from Simon Fraser University are dependent on the member's future salaries, it is necessary for a going-concern valuation to make an assumption regarding the future increases in such earnings. A member's salaries are assumed to increase with the rate of general earnings increases (4.25\% per annum for previous valuation; and 4.00\% per annum for current valuation). The salary scale was decreased since the previous valuation due to salaries increasing less than expected.

\vspace{3mm}
\textbf{YMPE Increase}

As the benefits paid to a member from Simon Fraser University are dependent on the future YMPE, it is necessary to make an assumption regarding the future increases in the YMPE. The YMPE was assumed to increase, up until the time the member retires, dies or terminates from active employment, at the assumed rate of 3\% per annum.

\vspace{3mm}
The YMPE for the next year is released by the Canadian Revenue Agency during November. For instance, this valuation uses a value of \$52,500 for the 2014 YMPE rather than projecting the 2013 YMPE at 3\%.

\vspace{3mm}
The same assumption was used in the previous valuation.

\section{Going-Concern Actuarial Cost Method}

An actuarial cost method is a technique used to allocate in a systematic and consistent manner the expected cost of a pension plan over the years of service during which plan members earn benefits under the plan. By pre-funding the cost of a pension plan in an orderly and rational manner, the security of benefits provided under the terms of the plan in respect of service that has already been rendered is significantly enhanced.

\vspace{3mm}
The Projected Unit Credit Actuarial Cost Method has been used for this valuation. Under this method, the actuarial present value of benefits in respect of service prior to the valuation date, but based on pensionable earnings projected to retirement, is compared with the actuarial asset value, revealing either a surplus or an unfunded actuarial liability.

\vspace{3mm}
The previous valuation also used the Projected Unit Credit Actuarial Cost Method.

\section{Going-Concern Asset Valuation Method}

For purposes of the going-concern valuation, assets have been valued at market value.

\mychapter{8}{Appendix D: Methods and Assumptions - Solvency}
The Canadian Institute of Actuaries requires to report the financial position of a pension plan on the assumption that the plan is wound-up on the effect date of the valuation, with benefits determined on the assumption that the pension plan has neither a surplus or a deficit.

\vspace{3mm}
The \textit{Income Tax Act} requires that a plan's solvency valuation liabilities be determined with the presumptions that:
\begin{itemize}
\item The plan is terminated and wound-up on the valuation date; and
\item The plan's liabilities are settled immediately
\end{itemize}

\vspace{3mm}
The following summarizes the prescribed assumptions, methods and benefits that make up the solvency basis for SFU at the valuation date. Judgment must be exercised in setting certain assumptions, especially as related to determining:
\begin{itemize}
\item The proportion of SFU's benefits expected to be settled by way of annuity purchase and by way of lump sum transfer; and
\item The hypothetical annuity purchase rates at the valuation date
\end{itemize}

\vspace{3mm}
Consequently, if SFU was terminated and settled on the valuation date, these solvency liabilities may be different than SFU's actual termination liabilities. Such difference may be primarily attributed to:
\begin{itemize}
\item Differences between the actual and assumed proportion of benefits being settled by annuity purchased and lump sum transfer; and 
\item An actual annuity purchase rate that is different than the rates assumed to be representative of the annuity market on the valuation date
\end{itemize}

\vspace{3mm}
If a shortfall exists on the solvency basis, this is amortized over 5 years.

\pagebreak
\section{Solvency Assumptions}
\vspace{3mm}
\bgroup
\normalsize
\def\arraystretch{1.5}
\begin{tabular}{p{6cm} p{4cm} p{4cm}}
\hline
 & \textbf{December 31, 2013} & \textbf{December 31, 2012} \\ \hline
Lump Sum  &  \multicolumn{2}{p{8cm}}{\tabitem 70\% of active members under age 55} \\ 
& \multicolumn{2}{p{8cm}}{\tabitem 50\% of active members over age 55}  \\
Annuity Purchase & \multicolumn{2}{p{8cm}}{All remaining members are assumed to elect to receive their benefit entitlement in the form of deferred or immediate pension. These benefits are assumed to be settled through the purchase of deferred or immediate annuities from a life insurance company.} \\ \hline
Interest Rates & \tabitem 3.25\% for CV & \tabitem 2.50\% for CV \\
& \tabitem 3.50\% for AP & \tabitem 3.00\% for AP \\ \hline
Mortality Rates & UP94 (to 2020) & UP94 (to 2015) \\ \hline
Retirement Age & \multicolumn{2}{p{8cm}}{Assumed members are retired at age 55 if under age 55. Otherwise, members are retired immediately.} \\ \hline
Wind-up Expense & \$325,000 & \$300,000 \\ \hline
\end{tabular} 
\egroup


\section{Solvency Actuarial Cost Method}

The solvency liabilities have been calculated as the actuarial present value of the benefits to which a member would be entitled if participation in SFU was terminated on the valuation date. The solvency actuarial cost method is based on the Traditional Unit Credit method for this valuation and the last valuation. It is further noted that the solvency liabilities do not take into consideration any benefit reductions that may be required in the event of actual SFU termination on the valuation date.

\section{Solvency Asset Valuation Method}

For purposes of the solvency valuation, assets have been valued at market value.

\section{Incremental Cost on a Solvency Basis}

The incremental cost on a solvency basis represents the present value, at the calculation date (time 0), of the expected aggregate change in the solvency liability between time 0 and the next calculation date (time t), adjusted upwards for the expected benefit payments between time 0 and time t.

\vspace{3mm}
An educational note was published in December 2010 by the Canadian Institute of Actuaries Committee of Pension Plan Financial Reporting to provide guidance for actuaries on the calculation of this information.

The calculation methodology can be summarized as follows:
\begin{itemize}
\item The present value at time 0 of expected benefit payments between time 0 and time t, discounted to time 0,

\textit{plus}
\item A projected hypothetical wind-up or solvency liability at time t, discounted to time 0, allowing for, if applicable to the pension plan being valued:
\begin{itemize}
\item Expected decrements and related changes in membership status between time 0 and time t
\item Accrual of service to time t
\item Expected changes in benefits to time t
\item A projection of pensionable earnings to time t
\end{itemize} 

\textit{minus}
\item The hypothetical wind-up or solvency liability at time 0
\end{itemize}

\vspace{3mm}
The projection calculations take into account the following assumptions and additional considerations:
\begin{itemize}
\item The assumptions for the expected benefits payments and decrement probabilities, service accruals, and projected changes in benefits and/or pensionable earnings that would be consistent with the assumptions used in the pension plan's going-concern valuation
\item The assumptions used to calculate the projected liability at time 0, assuming that interest rates remain at the levels applicable to time 0, that the select period is reset at time t for interest rate assumptions that are select and ultimate and that the Standards of Practice for the calculation of commuted values and the guidance for estimated annuity purchase costs in effect at time 0 remain in effect at time t
\item Active and inactive plan members as of time 0 and assumed new entrants over the period between time 0 and time t are considered in calculating the incremental cost
\end{itemize}

\mychapter{9}{Appendix E: Summary of Plan Provisions}

This valuation is based on the plan provisions in effect on December 31, 2013. The following is a brief summary of the provisions of SFU that affect costs and liabilities as at the valuation date. This summary reflects all SFU amendments up to the valuation date. This summary does not constitute a legal interpretation of SFU.

\vspace{3mm}
\bgroup
\normalsize
\def\arraystretch{1.5}
\begin{tabular}{p{6cm} p{4.5cm} p{4.5cm}}
\hline
Plan Effective Date & \textbf{December 31, 2013} & \textbf{December 31, 2012} \\ \hline
Eligibility for Membership  &  \multicolumn{2}{p{9cm}}{Eligible participants include full-time and part-time employees who meet criteria specified in the plan} \\  \hline
Pensionable Service & \multicolumn{2}{p{9cm}}{As defined under the provisions of the plan, shall not exceed 35 years} \\ \hline
Retirement Dates \\
\hspace{3mm} Normal Retirement Date & \multicolumn{2}{c}{Age 65}\\ 
\hspace{3mm} Early Retirement Date & \multicolumn{2}{c}{Age 60}\\ \hline
Normal Retirement Pension & 1.3\% of final average annual salary over 3 years (FAE3) up to final average YMPE over 3 years (FAYMPE3) multiplied by years of service,
 
 \vspace{3mm}
\textit{plus}

2.0\% of FAE3 in excess of FAYMPE3, multiplied by years of pensionable service 

& 1.3\% of final average annual salary over 5 years (FAE5) up to final average YMPE over 5 years (FAYMPE5), multiplied by years of service,

 \vspace{3mm}
\textit{plus}

2.0\% of FAY5 in excess of FAYMPE5, multiplied by years of pensionable service \\ \hline
Normal Form & \multicolumn{2}{c}{Single Life Pension} \\ \hline
Early Retirement Reduction & \multicolumn{2}{p{9cm}}{A member who has attained age 55 and who has completed two years of service, but whose age and years of pensionable service do not total 85, may elect to receive a pension commencing immediately that is reduced by an early retirement reduction} \\ 
& The early retirement reduction is 3\% per year from age 65 to the age at retirement if less than 65& The early retirement reduction is actuarial equivalent to the present value of benefits at age 65 \\ \hline
Death Benefit & \multicolumn{2}{p{9cm}}{If a member dies before retirement, benefits payable from the plan will be 100\% of the present value of accrued benefits to the date of death. Payments stop when the pension's guarantee period is over}  \\ \hline
Termination Benefits & \multicolumn{2}{p{9cm}}{If a member dies before retirement, benefits payable from the plan will be 100\% of the present value of accrued benefits to the date of termination of employment} \\ \hline
\end{tabular} 
\egroup


%%%% END %%%%
\end{document}
